
%%%%%%%%%%%%%%%%%%%%%% Feynman diagram for T2bbHH

\documentclass{article}

\input{shared/header.tex}

%%%%%%%%%%%%%%%%%%%%%%%%%%% Document %%%%%%%%%%%%%%%%%%%%%%%%%%%
\begin{document}
\thispagestyle{empty}


%%%%%%%% THE NAME OF THE fmffile HAS TO BE ``Feynman<filename>'' TO USE compile.py %%%%%%%%%%%%%%%
\begin{fmffile}{FeynmanT10StyleStealthGluino}
\parbox{300mm}{

\begin{fmfgraph*}(270,135) % \fmfpen{thick}
  \fmfstraight
  \fmfset{arrow_len}{cm}\fmfset{arrow_ang}{0}
  
  %%%%%%%%%%%% Specifying number of inputs/outputs
  \fmfleftn{i}{2}

  \fmfforce{0.w,0.95h}{i1}
  \fmfforce{0.w,0.05h}{i2}

  %% \fmfbottom{olantiq,olgamma,olgravitino,olgluino2}
  %% \fmfright{olgluino1,ougluino1}
  %% \fmftop{ouq,ougamma,ougravitino,ougluino2}

  \fmflabel{}{i1}
  \fmflabel{}{i2}
    
  %%%%%%%%%%%% Incoming protons (one line)
  % v1 is the main interaction vertex
  \fmf{fermion, tension=2., lab=p, label.side=right}{v1,i1}
  \fmf{fermion, tension=2., lab=p, label.side=left}{v1,i2}

  % set positions of upper output vertices manually
  \fmfforce{0.36w, h}{ouq}
  \fmfforce{0.52w, h}{ouantiq}
  \fmfforce{0.68w, h}{ougamma}
  \fmfforce{0.84w, h}{ougravitino}
  \fmfforce{w, h}{ougluino2}
  \fmfforce{w,0.75h}{ougluino1}

  % set positions of upper internal vertices manually
  \fmfforce{0.3w,0.75h}{v2}
  \fmfforce{0.52w,0.75h}{v3}
  \fmfforce{0.68w,0.75h}{v4}
  \fmfforce{0.84w,0.75h}{v5}

  % set positions of lower output vertices manually
  \fmfforce{0.36w, 0}{olq}
  \fmfforce{0.52w, 0}{olantiq}
  \fmfforce{0.68w, 0}{olgamma}
  \fmfforce{0.84w, 0}{olgravitino}
  \fmfforce{w, 0}{olgluino2}
  \fmfforce{w, 0.25h}{olgluino1}

  % set positions of lower internal vertices manually
  \fmfforce{0.3w,0.25h}{v6}
  \fmfforce{0.52w,0.25h}{v7}
  \fmfforce{0.68w,0.25h}{v8}
  \fmfforce{0.84w,0.25h}{v9}
         
  %%%%%%%%%%%% Produced SUSY particles
  \fmf{gluon, label=\sGlu, label.side=left, label.dist=3.}{v1,v2} % upper gluino
  \fmf{gluon, label=\sGlu, label.side=right, label.dist=3.}{v1,v6} % lower gluino
  % add straight line to the gluon pattern
  \fmf{fermion}{v1,v2} % upper gluino
  \fmf{fermion}{v1,v6} % lower gluino

  %%%%%%%%%%%%% Decays and vertex circles

  % upper gluino decay: neutralino
  \fmf{dots, label=\chiz,label.dist=-3, label.side=right, label.dist=3.}{v2,v3}
  % upper gluino decay: quark
  \fmf{fermion}{ouq,v2}
  \fmflabel{q}{ouq}
  % upper gluino decay: antiquark
  \fmf{fermion}{ouantiq,v2}
  \fmflabel{\anti{q}}{ouantiq}

  % upper neutralino decay: singlino
  \fmf{fermion, foreground=black, label=\PSS, label.side=right, label.dist=3.}{v3,v4}
  % upper neutralino decay: photon
  \fmf{photon}{ougamma,v3}
  \fmflabel{$\gamma$}{ougamma}

  % upper singlino decay: singlet
  \fmf{dashes, foreground=black, label=\PS, label.side=right, label.dist=3.}{v4,v5}
  % upper singlino decay: gravitino
  \fmf{fermion, foreground=black}{ougravitino,v4}
  \fmflabel{\sGra}{ougravitino}

  % upper singlet decay: lower gluon
  \fmf{gluon}{ougluino1,v5}
  \fmflabel{g}{ougluino1}
  % upper singlet decay: upper gluon
  \fmf{gluon}{ougluino2,v5}
  \fmflabel{g}{ougluino2}

  % lower gluino decay: neutralino
  \fmf{dots, label=\chiz,label.dist=+3, label.side=left, label.dist=3.}{v6,v7}
  % lower gluino decay: quark
  \fmf{fermion}{olq,v6}
  \fmflabel{q}{olq}
  % lower gluino decay: antiquark
  \fmf{fermion}{olantiq,v6}
  \fmflabel{\anti{q}}{olantiq}

  % lower neutralino decay: singlino
  \fmf{fermion, foreground=black, label=\PSS, label.side=left, label.dist=4.}{v7,v8}
  % lower neutralino decay: photon
  \fmf{photon}{olgamma,v7}
  \fmflabel{$\gamma$}{olgamma}

  % lower singlino decay: singlet
  \fmf{dashes, foreground=black, label=\PS, label.side=left, label.dist=3.}{v8,v9}
  % lower singlino decay: gravitino
  \fmf{fermion, foreground=black}{olgravitino,v8}
  \fmflabel{\sGra}{olgravitino}

  % lower singlet decay: upper gluon
  \fmf{gluon}{olgluino1,v9}
  \fmflabel{g}{olgluino1}
  % lower singlet decay: lower gluon
  \fmf{gluon}{olgluino2,v9}
  \fmflabel{g}{olgluino2}

               
  %% Vertex circles
  \fmfdot{v2,v3,v4,v5,v6,v7}
           
  %%%%%%%%%%%% Additional lines on incoming protons and blob
  %%%%%%%%%%%% Additional lines on incoming protons and blob
  \fmffreeze
  \renewcommand{\P}[3]{\fmfi{plain}{%
      vpath(__#1,__#2) shifted (thick*(#3))}}
  \P{i1}{v1}{2.,-0}
  \P{i1}{v1}{-2,0}
  \P{i2}{v1}{2.,0}
  \P{i2}{v1}{-2.,-0}
  \fmfv{decor.shape=circle,decor.filled=30, decor.size=.12w}{v1}
  
\end{fmfgraph*}
       
}           
\end{fmffile} 

\end{document}
