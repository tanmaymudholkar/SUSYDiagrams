
%%%%%%%%%%%%%%%%%%%%%% Feynman diagram for T5qqqqGG

\documentclass{article}

\input{shared/header.tex}


%%%%%%%%%%%%%%%%%%%%%%%%%%% Document %%%%%%%%%%%%%%%%%%%%%%%%%%%
\begin{document}
\thispagestyle{empty}


%%%%%%%% THE NAME OF THE fmffile HAS TO BE ``Feynman<filename>'' TO USE compile.py %%%%%%%%%%%%%%%
\begin{fmffile}{FeynmanT5qqqqGGStealth}
\parbox{600mm}{

\begin{fmfgraph*}(360,270) %% \fmfpen{thick}

  \fmfset{arrow_len}{cm}\fmfset{arrow_ang}{0}

  %%%%%%%%%%%% Specifying number of inputs/outputs
  \fmfleftn{i}{2}
  \fmfrightn{o}{12}
  \fmflabel{}{i1}
  \fmflabel{}{i2}

  %%%%%%%%%%%% Incoming protons (one line)
  \fmf{fermion, tension=10, lab=p, label.side=right}{v1,i1}
  \fmf{fermion, tension=10, lab=p, label.side=left}{v1,i2}

  %%%%%%%%%%%% Produced SUSY particles
  \fmf{gluon, tension=10, label=\sGlu, label.side=right, label.dist=10.}{v1,v2} % lower gluino
  \fmf{gluon, tension=10, label=\sGlu, label.side=left, label.dist=8.}{v1,v3} % upper gluino
  % add straight line to the gluon pattern
  \fmf{fermion}{v1,v2} % lower gluino
  \fmf{fermion}{v1,v3} % upper gluino

  %%%%%%%%%%%%% Decays and vertex circles

  % lower gluino decay: quark
  \fmf{fermion}{v2,o1}
  \fmflabel{q}{o1}

  % lower gluino decay: antiquark
  \fmf{fermion}{v2,o2}
  \fmflabel{\anti{q}}{o2}

  % lower gluino decay: neutralino
  \fmf{dots, tension=8, foreground=black, label=\chiz, label.side=left, label.dist=2}{v2,v4}

  % upper gluino decay: quark
  \fmf{fermion}{v3,o12}
  \fmflabel{q}{o12}

  % upper gluino decay: antiquark
  \fmf{fermion}{v3,o11}
  \fmflabel{\anti{q}}{o11}

  % upper gluino decay: neutralino
  \fmf{dots, tension=8, foreground=black, label=\chiz, label.side=right, label.dist=2}{v3,v5}

  % lower neutralino decay: photon
  \fmf{photon}{v4,o6}
  \fmflabel{$\gamma$}{o6}

  % lower neutralino decay: singlino
  \fmf{fermion, tension=4, foreground=black, label=\PSS, label.side=right, label.dist=2}{v4,v6}

  % upper neutralino decay: photon
  \fmf{photon}{v5,o7}
  \fmflabel{$\gamma$}{o7}

  % upper neutralino decay: singlino
  \fmf{fermion, tension=4, foreground=black, label=\PSS, label.side=left, label.dist=2}{v5,v7}

  % lower singlino decay: gravitino
  \fmf{fermion, foreground=black}{v6,o5}
  \fmflabel{\sGra}{o5}

  % lower singlino decay: singlet
  \fmf{dashes, tension=3, foreground=black, label=\PS, label.side=right, label.dist=2}{v6,v8}

  % upper singlino decay: gravitino
  \fmf{fermion, foreground=black}{v7,o8}
  \fmflabel{\sGra}{o8}

  % upper singlino decay: singlet
  \fmf{dashes, tension=3, foreground=black, label=\PS, label.side=left, label.dist=2}{v7,v9}

  % lower singlet decay: lower gluon
  \fmf{gluon}{v8,o3}
  \fmflabel{g}{o3}

  % lower singlet decay: upper gluon
  \fmf{gluon}{v8,o4}
  \fmflabel{g}{o4}

  % upper singlet decay: lower gluon
  \fmf{gluon}{v9,o9}
  \fmflabel{g}{o9}

  % upper singlet decay: upper gluon
  \fmf{gluon}{v9,o10}
  \fmflabel{g}{o10}

  %% Vertex circles
  \fmfdot{v2,v3,v4,v5,v6,v7,v8,v9}

  %%%%%%%%%%%% Additional lines on incoming protons and blob
  \input{shared/protons.tex}


\end{fmfgraph*}

}
\end{fmffile}

\end{document}
